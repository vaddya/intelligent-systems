\documentclass[a4paper,14pt]{extarticle}

\usepackage[utf8x]{inputenc}
\usepackage[T1]{fontenc}
\usepackage[russian]{babel}
\usepackage{hyperref}
\usepackage{indentfirst}
\usepackage{here}
\usepackage{array}
\usepackage{graphicx}
\usepackage{caption}
\usepackage{subcaption}
\usepackage{chngcntr}
\usepackage{amsmath}
\usepackage{amssymb}
\usepackage[left=2cm,right=2cm,top=2cm,bottom=2cm,bindingoffset=0cm]{geometry}
\usepackage{multicol}
\usepackage{multirow}
\usepackage{titlesec}
\usepackage{listings}
\usepackage{listingsutf8}
\usepackage{color}
\usepackage{enumitem}
\usepackage{cmap}
\usepackage{url}

\definecolor{green}{rgb}{0,0.6,0}
\definecolor{gray}{rgb}{0.5,0.5,0.5}
\definecolor{purple}{rgb}{0.58,0,0.82}

\lstdefinelanguage{none}{}

\lstset{
	language={Prolog},
	inputpath={../},
	backgroundcolor=\color{white},
	commentstyle=\color{green},
	keywordstyle=\color{blue},
	numberstyle=\color{gray}\scriptsize\ttfamily,
	stringstyle=\color{purple},
	basicstyle=\lst@ifdisplaystyle\footnotesize\fi\ttfamily,
	breakatwhitespace=false,
	breaklines=true,
	captionpos=b,
	keepspaces=true,
	numbers=left,
	numbersep=5pt,
	showspaces=false,
	showstringspaces=false,
	showtabs=false,
	tabsize=4,
	frame=single,
	morekeywords={implement, open, domains, class, predicates, clauses, string, symbol, real, in, out, nondeterm, anyflow, end, goal},
	deletekeywords={},
	extendedchars=false,
	columns=fullflexible,
	inputencoding=utf8/cp1251,
	literate=%
		{~}{{\raise.25ex\hbox{$\mathtt{\sim}$}}}{1}
}

\titleformat*{\section}{\large\bfseries} 
\titleformat*{\subsection}{\normalsize\bfseries} 
\titleformat*{\subsubsection}{\normalsize\bfseries} 
\titleformat*{\paragraph}{\normalsize\bfseries} 
\titleformat*{\subparagraph}{\normalsize\bfseries} 

\counterwithin{figure}{section}
\counterwithin{equation}{section}
\counterwithin{table}{section}
\newcommand{\sign}[1][5cm]{\makebox[#1]{\hrulefill}}
\newcommand{\equipollence}{\quad\Leftrightarrow\quad}
\newcommand{\no}[1]{\overline{#1}}
\newcommand{\code}[1]{\lstinline[language=none]|#1|}
\newcommand{\exsys}[1]{
	\lstinputlisting[caption=\code{#1/main.pro}]{src/#1/main.pro}
	Запустим программу \code{#1}:
	\lstinputlisting[language=none]{logs/#1.txt}
}

\graphicspath{{../pics/}}
\captionsetup{justification=centering,margin=1cm}
\def\arraystretch{1.3}
\setlength\parindent{5ex}
\titlelabel{\thetitle.\quad}

\setitemize{topsep=0em, itemsep=0em}
\setenumerate{topsep=0em, itemsep=0em}

\begin{document}

\begin{titlepage}
\begin{center}
	Санкт-Петербургский Политехнический Университет Петра Великого\\[0.3cm]
	Институт компьютерных наук и технологий \\[0.3cm]
	Кафедра компьютерных систем и программных технологий\\[4cm]
	
	\textbf{ОТЧЕТ}\\ 
	\textbf{по лабораторной работе}\\[0.5cm]
	\textbf{<<Разработка простой интеллектуальной системы на языке PROLOG>>}\\[0.1cm]
	Интеллектуальные системы\\[3.0cm]
\end{center}

\begin{flushright}
	\begin{minipage}{0.5\textwidth}
		\textbf{Работу выполнил студент}\\[3mm]
		гр. 3540901/91502 \hfill \sign[1.1cm] \hfill Дьячков В.В.\\[5mm]
		\textbf{Работу принял преподаватель}\\[5mm]
		\sign[2.1cm] \hfill к.т.н., доц. Бендерская Е.Н. \\[5mm]
	\end{minipage}
\end{flushright}

\vfill

\begin{center}
	Санкт-Петербург\\[0.3cm]
	\the\year
\end{center}
\end{titlepage}

\addtocounter{page}{1}

\tableofcontents
\newpage

\section{Программа работы}

\begin{enumerate}
	\item Изучить алгоритм кластеризации \code{DBSCAN}.
	\item Выделить параметры алгоритма.
	\item Написать программу кластеризации на базе заданного алгоритма.
	\item Проверить работоспособность на тестовых примерах базы \cite{data}. Привести результаты работы программы (таблицы и графики) при разных значениях параметров алгоритма для всех тестовых данных.
	\item Описать достоинства и недостатки алгоритма, а также способы модификаций и развития для улучшения качества работы.
	\item Привести альтернативные способы решения задачи кластеризации.
\end{enumerate}

\section{Выполнение работы}

\subsection{Алгоритм кластеризации \code{DBSCAN}}

Основанная на плотности пространственная кластеризация для приложений с шумами (англ. Density-based spatial clustering of applications with noise, DBSCAN) -- это алгоритм кластеризации данных, который предложили Маритин Эстер, Ганс-Петер Кригель, Ёрг Сандер и Сяовэй Су в 1996. Это алгоритм кластеризации, основанной на плотности -- если дан набор точек в некотором пространстве, алгоритм группирует вместе точки, которые тесно расположены (точки со многими близкими соседями), помечая как выбросы точки, которые находятся одиноко в областях с малой плотностью (ближайшие соседи которых лежат далеко). DBSCAN является одним из наиболее часто используемых алгоритмов кластеризации, и наиболее часто упоминается в научной литературе \cite{wiki}.

\subsection{Параметры алгоритма \code{DBSCAN}}

DBSCAN требует задания двух параметров:

\begin{itemize}
	\item \code{eps} ($\epsilon$) -- размер окрестности, в рамках которой будут рассматриваться другие точки для определения принадлежности к кластеру текущей точки.
	\item \code{min_samples} ($m$) -- минимальное число вокруг точки, которые образуют с ней плотную область.
\end{itemize}

Соотношение $\alpha = \dfrac{m}{\epsilon^n}$, где $n$ -- размерность пространства, можно интуитивно рассматривать как пороговую плотность точек данных в области пространства. Ожидаемо, что при одинаковом соотношении $\alpha$, и результаты будут примерно одинаковы. Иногда это действительно так, но есть причина, почему алгоритму нужно задать два параметра, а не один. Во-первых типичное расстояние между точками в разных датасетах разное — явно задавать радиус приходится всегда. Во-вторых, играют роль неоднородности датасета. Чем больше $m$ и $\epsilon$, тем больше алгоритм склонен «прощать» вариации плотности в кластерах. С одной стороны, это может быть полезно: неприятно увидеть в кластере «дырки», где просто не хватило данных. С другой стороны, это вредно, когда между кластерами нет чёткой границы или шум создаёт «мост» между скоплениями. Тогда DBSCAN запросто соединит две разные группы. В балансе этих параметров и кроется сложность применения DBSCAN: реальные наборы данных содержат кластеры разной плотности с границами разной степени размытости \cite{habr}.

\subsection{Реализация алгоритма}

\subsection{Результаты работы на тестовых примерах}

\subsection{Достоинства, недостатки и улучшения}

\subsection{Альтернативные алгоритмы}

\section{Выводы}

\bibliographystyle{plain}
\addcontentsline{toc}{section}{Список литературы}
\bibliography{refs}

\end{document}
