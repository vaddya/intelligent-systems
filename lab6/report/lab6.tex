\documentclass[a4paper,14pt]{extarticle}

\usepackage[utf8x]{inputenc}
\usepackage[T1]{fontenc}
\usepackage[russian]{babel}
\usepackage{hyperref}
\usepackage{indentfirst}
\usepackage{here}
\usepackage{array}
\usepackage{graphicx}
\usepackage{caption}
\usepackage{subcaption}
\usepackage{chngcntr}
\usepackage{amsmath}
\usepackage{amssymb}
\usepackage[left=2cm,right=2cm,top=2cm,bottom=2cm,bindingoffset=0cm]{geometry}
\usepackage{multicol}
\usepackage{multirow}
\usepackage{titlesec}
\usepackage{listings}
\usepackage{color}
\usepackage{enumitem}
\usepackage{cmap}
\usepackage{url}

\definecolor{green}{rgb}{0,0.6,0}
\definecolor{gray}{rgb}{0.5,0.5,0.5}
\definecolor{purple}{rgb}{0.58,0,0.82}

\lstdefinelanguage{none}{}

\lstset{
	language={none},
	inputpath={../logs/},
	backgroundcolor=\color{white},
	commentstyle=\color{green},
	keywordstyle=\color{blue},
	numberstyle=\color{gray}\scriptsize\ttfamily,
	stringstyle=\color{purple},
	basicstyle=\footnotesize\ttfamily,
	breakatwhitespace=false,
	breaklines=true,
	captionpos=b,
	keepspaces=true,
	numbers=left,
	numbersep=5pt,
	showspaces=false,
	showstringspaces=false,
	showtabs=false,
	tabsize=4,
	frame=single,
	morekeywords={},
	deletekeywords={},
	extendedchars=false,
	columns=fullflexible,
	literate=%
		{~}{{\raise.25ex\hbox{$\mathtt{\sim}$}}}{1}%
		{-}{-}{1}
}

\titleformat*{\section}{\large\bfseries} 
\titleformat*{\subsection}{\normalsize\bfseries} 
\titleformat*{\subsubsection}{\normalsize\bfseries} 
\titleformat*{\paragraph}{\normalsize\bfseries} 
\titleformat*{\subparagraph}{\normalsize\bfseries} 

\counterwithin{figure}{section}
\counterwithin{equation}{section}
\counterwithin{table}{section}
\newcommand{\sign}[1][5cm]{\makebox[#1]{\hrulefill}}
\newcommand{\equipollence}{\quad\Leftrightarrow\quad}
\newcommand{\no}[1]{\overline{#1}}
\graphicspath{{../pics/}}
\captionsetup{justification=centering,margin=1cm}
\def\arraystretch{1.3}
\setlength\parindent{5ex}
\titlelabel{\thetitle.\quad}

\setitemize{topsep=0em, itemsep=0em}
\setenumerate{topsep=0em, itemsep=0em}

\begin{document}

\begin{titlepage}
\begin{center}
	Санкт-Петербургский Политехнический Университет Петра Великого\\[0.3cm]
	Институт компьютерных наук и технологий \\[0.3cm]
	Кафедра компьютерных систем и программных технологий\\[4cm]
	
	\textbf{ОТЧЕТ}\\ 
	\textbf{по лабораторной работе}\\[0.5cm]
	\textbf{<<Data Mining>>}\\[0.1cm]
	Интеллектуальные системы\\[3.0cm]
\end{center}

\begin{flushright}
	\begin{minipage}{0.5\textwidth}
		\textbf{Работу выполнил студент}\\[3mm]
		гр. 3540901/91502 \hfill \sign[1.1cm] \hfill Дьячков В.В.\\[5mm]
		\textbf{Работу принял преподаватель}\\[5mm]
		\sign[2.1cm] \hfill к.т.н., доц. Бендерская Е.Н. \\[5mm]
	\end{minipage}
\end{flushright}

\vfill

\begin{center}
	Санкт-Петербург\\[0.3cm]
	\the\year
\end{center}
\end{titlepage}

\addtocounter{page}{1}

\tableofcontents
\newpage

\section{Программа работы}

\begin{enumerate}
	\item Выполнить 10 вариант из индивидуальных заданий (см. задание 9-15 на стр. 32-34 из пособия Бураков С. В. «Язык логического программирования PROLOG», СПбГУАП, 2003 \cite{burakov}).
	\item Изучить 1-2 лабы по методичке - Середа С.Н. «Методичка по языку Prolog», Муромский университет. 2003г \cite{sereda}.
	\item Решить задачу с помощью Prolog.
	\item В выводах отразить, помимо своих мыслей, возникших в ходе работы, ответы на приведенные ниже вопросы:
	\begin{enumerate}
		\item В чем плюсы и минусы языка Prolog?
		\item Какие еще языки используются для разработки ИИ, приведите примеры (НЕ МЕНЕЕ 2-х) проектов, языков и краткое описание проектов. (Альтернативы PROLOG)
		\item Решаема ли проблема комбинаторного взрыва, пути решения?
		\item Корректно ли по-вашему в принципе разработка языка ИИ? Что он должен из себя представлять?
		\item Можно ли разработать ИИ не понимая, как он работает, должны ли мы понимать, как он работает, думает, рассуждает? 
	\end{enumerate}
\end{enumerate}

\section{Выполнение работы}

\subsection{Индивидуальное задание}

\paragraph{Задача.} Витя, Юра и Миша сидели на скамейке. В каком порядке они
сидели, если известно, что Юра сидел слева от Миши и справа от Вити.

\prolog{task1}

Полученное решение удовлетворяет условиям задачи.

\subsection{Решение задачи}

\paragraph{Задача.} Как-то раз случай свёл в купе астронома, поэта, прозаика и драматурга. Это были Алексеев, Борисов, Константинов и Дмитриев. Оказалось, что каждый из них взял с собой книгу, написанную одним из пассажиров этого купе. Алексеев и Борисов углубились в чтение предварительно обменявшись книгами. Поэт читал пьесу, прозаик — очень молодой человек, выпустивший свою книгу, говорил, что он никогда и ничего не читал по астрономии. Борисов купил одно из произведений Дмитриева. Никто из пассажиров не читал свои книги. Что читал каждый из них, кто кем был?

\prolog{task2}

Видно, что было получено два решения, которые удовлетворяют условию задачи.

\section{Дополнительные вопросы}

\begin{enumerate}
	\item В чем плюсы и минусы языка Prolog?
	\begin{itemize}
		\item[$+$] Возможность решения логических задач.
		\item[$-$] Много разных \textbf{несовместимых} друг с другом реализаций языка.
		\item[$-$] Неудобно описывать арифметические операции.
		\item[$-$] Низкая популярность языка (50 место в рейтинге TIOBE).
	\end{itemize}
	
	\item Какие еще языки используются для разработки ИИ, приведите примеры (НЕ МЕНЕЕ 2-х) проектов, языков и краткое описание проектов.
	\begin{itemize}
		\item \textbf{Язык Datalog.} Данный язык дедуктивных запросов является подмножеством языка логического программирования Prolog. Однако программа на языке Datalog полностью декларативна, в то время как программы на языке Prolog на практике имеют двойное значение — декларативное (логическое) и императивное (процедурное).
		\item \textbf{Язык Python.} Язык общего назначения, обладающим огромным количеством библиотек для ИИ (напр., нейронные сети).
	\end{itemize}
	
	\item Решаема ли проблема комбинаторного взрыва, пути решения?
	
	Комбинаторным взрывом называют резкое возрастание временной сложности алгоритма при увеличении объема входных данных. Проблема комбинаторного взрыва в общем случае не решена, однако, использование эвристик (например, градиентный бустинг) помогает значительно ускорить поиск решения, возможно пожертвовав его точностью.
	
	\item Корректно ли по-вашему в принципе разработка языка ИИ? Что он должен из себя представлять?
	
	В настоящее время уже существует множество библиотек, достаточных для разработки систем с ИИ (например, нейронных сетей). Данные инструменты представлены на различных языках программирования общего назначения, поэтому создание отдельного языка для описания ИИ может быть излишним. Язык для ИИ должен быть декларативным, кратким и простым в использовании.
	
	\item Можно ли разработать ИИ не понимая, как он работает, должны ли мы понимать, как он работает, думает, рассуждает?
	
	Понимать принцип работы при разработке систем ИИ необходимо, т.к. это позволяет осуществлять контроль и управление такими системами. В то же время, описание ИИ должно быть декларативным (без деталей реализации), на уровне использования основных высокоуровневых компонентов и ограничений.
	
\end{enumerate}

\section{Выводы}

В работе рассмотрен особенности языка Prolog, с помощью которого были решены две логические задачи. Исходный код программы и их результаты работы приведены в отчете. Сформулированы достоинства и недостатки языка Prolog, а также рассмотрены некоторые вопросы области искусственного интеллекта.


\bibliographystyle{plain}
\addcontentsline{toc}{section}{Список литературы}
\bibliography{refs}

\end{document}
