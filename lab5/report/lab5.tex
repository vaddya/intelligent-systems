\documentclass[a4paper,14pt]{extarticle}

\usepackage[utf8x]{inputenc}
\usepackage[T1]{fontenc}
\usepackage[russian]{babel}
\usepackage{hyperref}
\usepackage{indentfirst}
\usepackage{here}
\usepackage{array}
\usepackage{graphicx}
\usepackage{caption}
\usepackage{subcaption}
\usepackage{chngcntr}
\usepackage{amsmath}
\usepackage{amssymb}
\usepackage[left=2cm,right=2cm,top=2cm,bottom=2cm,bindingoffset=0cm]{geometry}
\usepackage{multicol}
\usepackage{multirow}
\usepackage{titlesec}
\usepackage{listings}
\usepackage{color}
\usepackage{enumitem}
\usepackage{cmap}
\usepackage{url}

\definecolor{green}{rgb}{0,0.6,0}
\definecolor{gray}{rgb}{0.5,0.5,0.5}
\definecolor{purple}{rgb}{0.58,0,0.82}

\lstdefinelanguage{none}{}

\lstset{
	language={none},
	inputpath={../logs/},
	backgroundcolor=\color{white},
	commentstyle=\color{green},
	keywordstyle=\color{blue},
	numberstyle=\color{gray}\scriptsize\ttfamily,
	stringstyle=\color{purple},
	basicstyle=\footnotesize\ttfamily,
	breakatwhitespace=false,
	breaklines=true,
	captionpos=b,
	keepspaces=true,
	numbers=left,
	numbersep=5pt,
	showspaces=false,
	showstringspaces=false,
	showtabs=false,
	tabsize=4,
	frame=single,
	morekeywords={},
	deletekeywords={},
	extendedchars=false,
	columns=fullflexible,
	literate=%
		{~}{{\raise.25ex\hbox{$\mathtt{\sim}$}}}{1}%
		{-}{-}{1}
}

\titleformat*{\section}{\large\bfseries} 
\titleformat*{\subsection}{\normalsize\bfseries} 
\titleformat*{\subsubsection}{\normalsize\bfseries} 
\titleformat*{\paragraph}{\normalsize\bfseries} 
\titleformat*{\subparagraph}{\normalsize\bfseries} 

\counterwithin{figure}{section}
\counterwithin{equation}{section}
\counterwithin{table}{section}
\newcommand{\sign}[1][5cm]{\makebox[#1]{\hrulefill}}
\newcommand{\equipollence}{\quad\Leftrightarrow\quad}
\newcommand{\no}[1]{\overline{#1}}
\graphicspath{{../pics/}}
\captionsetup{justification=centering,margin=1cm}
\def\arraystretch{1.3}
\setlength\parindent{5ex}
\titlelabel{\thetitle.\quad}

\setitemize{topsep=0em, itemsep=0em}
\setenumerate{topsep=0em, itemsep=0em}

\begin{document}

\begin{titlepage}
\begin{center}
	Санкт-Петербургский Политехнический Университет Петра Великого\\[0.3cm]
	Институт компьютерных наук и технологий \\[0.3cm]
	Кафедра компьютерных систем и программных технологий\\[4cm]
	
	\textbf{ОТЧЕТ}\\ 
	\textbf{по лабораторной работе}\\[0.5cm]
	\textbf{<<Data Mining>>}\\[0.1cm]
	Интеллектуальные системы\\[3.0cm]
\end{center}

\begin{flushright}
	\begin{minipage}{0.5\textwidth}
		\textbf{Работу выполнил студент}\\[3mm]
		гр. 3540901/91502 \hfill \sign[1.1cm] \hfill Дьячков В.В.\\[5mm]
		\textbf{Работу принял преподаватель}\\[5mm]
		\sign[2.1cm] \hfill к.т.н., доц. Бендерская Е.Н. \\[5mm]
	\end{minipage}
\end{flushright}

\vfill

\begin{center}
	Санкт-Петербург\\[0.3cm]
	\the\year
\end{center}
\end{titlepage}

\addtocounter{page}{1}

\tableofcontents
\newpage

\section{Программа работы}

\begin{enumerate}
	\item Получить начальное представление о синтаксисе и семантике базовых конструкций языка PROLOG, ознакомившись с разделами 1-5 методического пособия - Бураков С. В. <<Язык логического программирования PROLOG>>, СПбГУАП, 2003.
	\item Создать проект в оболочке Visual Prolog по аналогии с примером.
	\item Построить генеалогическое дерево для данного примера на основе результатов выполнения программы и исходного кода программы. (см. ход работы).
	\item Построить описание онтологии из данного примера на естественном языке.
	\item Построить концептуальную карту (семантическую сеть), описывающую данный пример.
	\item Создать проекты 1-21 для каждого из примеров в пособии из п.1 и привести листинги результатов работы каждой из программ в ответ на запросы пользователя.
\end{enumerate}

\newpage

\section{Выполнение работы}

\subsection{Компоненты экспертной системы}

\begin{itemize}
	\item Диалоговый компонент
	\item База данных
	\item База знаний
	\item Решатель
\end{itemize}

\section{Выполнение работы}

\subsection{Тестовой пример \code{family}}

\exsys{family}

\subsection{Примеры из пособия}

\subsubsection{Географические субъекты}

\exsys{project1}

\subsubsection{Составные объекты}

\exsys{project2}

\subsubsection{Задача о семейных отношениях}

\exsys{project3}

\subsubsection{Арифметические операции}

\exsys{project4}

\subsubsection{Использование анонимных переменных}

\exsys{project5}

\subsubsection{Явное указание цели}

\exsys{project6}

\subsubsection{Предикат \code{fail}}

\exsys{project7}

\subsubsection{Предикат \code{cut} (1)}

\exsys{project8}

\subsubsection{Предикат \code{cut} (2)}

\exsys{project9}

\subsubsection{Рекурсия}

\exsys{project10}

\subsubsection{Сумма цифр числа}

\exsys{project11}

\subsubsection{Ханойская башня}

\exsys{project12}

\subsubsection{Списки}

\exsys{project13}

\subsubsection{Правило \code{find_it}}

\exsys{project14}

\subsubsection{Сумма элементов списка}

\exsys{project15}

\subsubsection{Мужик, волк, коза и капуста}

\exsys{project16}

\subsubsection{Логическая задача (1)}

\exsys{project17}

\subsubsection{Логическая задача (2)}

\exsys{project18}

\subsubsection{Логическая задача (3)}

\exsys{project19}

\subsubsection{База данных}

\exsys{project20}

\subsubsection{Простая экспертная система}

\exsys{project21}

\section{Выводы}

\newpage

\bibliographystyle{plain}
\addcontentsline{toc}{section}{Список литературы}
\bibliography{refs}

\end{document}
